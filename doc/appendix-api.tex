\chapter{The Variables API File}
\index{appendix!API file}
\label{cha:api}
\minitoc

In this chapter all variables which may be found in a variables 
API file are explained. The variables API is normally created by SaX2's own 
configuration interface, but this is not absolutely necessary. As soon as a 
variables API exists, this can be used to create a configuration file. 
In conjunction with the \textit{ImportAPI} module and the
\textit{CreateSections} module, an X11 configuration can be created from the 
API file. 

\section{API File Keyword Explanations}
The individual tables in their format description use various keywords, which
are explained in the following list. 

\begin{itemize}
\item \textbf{String:}\\
Refers to any sequence of characters which are \textbf{not}
embedded in quotation marks. 
\item \textbf{Subsection:}\\
Refers to the name of a subsection in the X11 configuration.
This word is followed by an entry in the subsection.
\item \textbf{Flagname:}\\
Refers to the name of a server flag. This is followed by the value for the 
server flag.
\item \textbf{Integer:}\\
Refers to a whole number. Is usually used in connection with variables for
defining size. 
\item \textbf{ButtonX:}\\
Refers to the number of the mouse button which should be adapted for the wheel
movement in the X axis.
\item \textbf{ButtonY:}\\
Refers to the number of the mouse button which should be adapted for the wheel
movement in the Y axis.
\item \textbf{Clocks:}\\
Refers to a list separated by spaces. with clock values. These values can be
whole numbers as well as fractions. 
\item \textbf{Mode:}\\
Refers to a resolution string in the form of \verb+[Xpixel]x[Ypixel]+
\item \textbf{Algorithm:}\\
Refers to the two possible algorithms
\textit{CheckDesktopGeometry} or \textit{IteratePrecisely}.
\item \textbf{Modeline:}\\
Refers to a modeline string, starting with a name in quotation marks which
must match a \textit{resolution string}, followed by the RamDAC speed and 8
further parameters. 
\item \textbf{Sync:}\\
Refers to a frequency range. This is specified through a number range in the
format:  \verb+[Minimum]-[Maximum]+ 
\item \textbf{Left,Right,Up,Down}\\
Refers to a screen position. The value matches an identifier string in
accordance with the monitor. If there is no screen at this point then
\verb+<none>+ should be entered.
\end{itemize}

\section{API File Overview Tables of all Possible Variables}
Below, all variables which may appear in an API file are listed in tabular
form. The contents of each table refer to a section in the API file. It should
be noted that one API section may cover a number of xorg.conf sections.

%====================
% Pfad Variablen
%--------------------
\index{Section!files}
\index{Section!module}
\index{Section!serverflags}
\subsection{Path Variables: Section Files, Modules and Server Flags}
\begin{tabular}[h]{|p{2cm}|p{4cm}|p{7cm}|}
 \hline
 \textbf{Identifier} & \textbf{Variable}     & \textbf{Format}                \\
 \hline
 Integer & FontPath     & String,String,String,...                       \\
 Integer & RgbPath      & String,String,String,...                       \\ 
 Integer & ModulePath   & String,String,String,...                       \\
 Integer & ModuleLoad   & String,String,String,...                       \\
 Integer & Extmod       & Subsection,String\verb+\n+Subsectio,String,... \\ 
 Integer & SpecialFlags & Flagname,String\verb+\n+Flagname,String,...    \\
 Integer & ServerFlags  & String,String,String,...                       \\
 \hline
\end{tabular}

%==========================
% Graphikkarten Variablen
%--------------------------
\index{Section!Device}
\subsection{Card Variables: Device Section}
\begin{tabular}[h]{|p{2cm}|p{4cm}|p{7cm}|}
 \hline
 \textbf{Identifier} & \textbf{Variable}     & \textbf{Format}        \\
 \hline
 Integer & Identifier      & String                             \\
 Integer & Driver          & String                             \\
 Integer & Memory          & Integer                            \\
 Integer & BusID           & String                             \\
 Integer & Vendor          & String                             \\
 Integer & Name            & String                             \\
 Integer & DacChip         & String                             \\
 Integer & GraphicsChip    & String                             \\
 Integer & ClockChip       & String                             \\
 Integer & DacSpeed        & String                             \\
 Integer & Clocks          & Clocks,Clocks,...                  \\
 Integer & Option          & String,String,...                  \\
 Integer & RawData         & String,String,...                  \\
 Integer & MaxDac          & Integer                            \\
 \hline
\end{tabular}


%====================
% Maus Variablen
%--------------------
\index{Section!InputDevice}
\subsection{Mouse Variables: InputDevice Section}
\begin{tabular}[h]{|p{2cm}|p{4cm}|p{7cm}|}
 \hline
 \textbf{Identifier} & \textbf{Variable}     & \textbf{Format}        \\
 \hline
 Integer & Identifier       & String                             \\
 Integer & Driver           & String                             \\
 Integer & Protocol         & String                             \\
 Integer & Device           & String                             \\
 Integer & Baudrate         & Integer                            \\
 Integer & Samplerate       & Integer                            \\
 Integer & Emulate3Buttons  & Yes \verb+|+ No                    \\
 Integer & Emulate3Timeout  & Integer                            \\
 Integer & ChordMiddle      & Yes \verb+|+ No                    \\
 Integer & MinX             & Integer                            \\
 Integer & MaxX             & Integer                            \\
 Integer & MinY             & Integer                            \\
 Integer & MaxY             & Integer                            \\
 Integer & ScreenNumber     & Integer                            \\
 Integer & ReportingMode    & String                             \\
 Integer & ButtonNumber     & Integer                            \\
 Integer & ButtonThreshold  & Integer                            \\
 Integer & SendCoreEvents   & Yes \verb+|+ No                    \\
 Integer & ClearDTR         & Yes \verb+|+ No                    \\
 Integer & ClearRTS         & Yes \verb+|+ No                    \\
 Integer & ZAxisMapping     & Off \verb+|+ None \verb+|+ ButtonX ButtonY \verb+|+ X \verb+|+ Y \\
 Integer & ZAxisNegMove     & Off \verb+|+ ButtonX               \\
 Integer & ZAxisPosMove     & Off \verb+|+ ButtonY               \\
 Integer & Vendor           & String                             \\
 Integer & Name             & String                             \\
 Integer & TabletMode       & String                             \\
 Integer & TabletType       & String                             \\
 \hline
\end{tabular}

%=========================
% Oberflaechen Variablen
%-------------------------
\index{Section!Monitor}
\index{Section!Modes}
\index{Section!Screen}
\subsection{Desktop Variables: Section Monitor,Modes and Screen}
\begin{tabular}[h]{|p{2cm}|p{4cm}|p{7cm}|}
 \hline
 \textbf{Identifier} & \textbf{Variable}     & \textbf{Format}        \\
 \hline
 Integer & Identifier       & String                             \\  
 Integer & Device           & String                             \\
 Integer & Monitor          & String                             \\
 Integer & VendorName       & String                             \\
 Integer & ModelName        & String                             \\
 Integer & Virtual          & Integer Integer                    \\
 Integer & Visual           & String                             \\
 Integer & HorizSync        & Sync                               \\
 Integer & VertRefresh      & Sync                               \\
 Integer & MonitorOptions   & String,String,...                  \\
 Integer & ScreenOptions    & String,String,...                  \\
 Integer & Modelines        & Modeline,Modeline,...              \\
 Integer & SpecialModeline  & Modeline,Modeline,...              \\
 Integer & ColorDepth       & Integer                            \\
 Integer & CalcModelines    & Yes \verb+|+ No                    \\
 Integer & CalcAlgorithm    & Algorithm                          \\
 Integer & ViewPort         & Integer Integer                    \\
 Integer & ScreenRawLine    & String,String,...                  \\
 Integer & Modes:4          & Mode,Mode,...                      \\
 Integer & Modes:8          & Mode,Mode,...                      \\
 Integer & Modes:15         & Mode,Mode,...                      \\
 Integer & Modes:16         & Mode,Mode,...                      \\
 Integer & Modes:24         & Mode,Mode,...                      \\
 Integer & Modes:32         & Mode,Mode,...                      \\
 Integer & ImportXFineCache & Yes \verb+|+ No                    \\
 \hline
\end{tabular}

%====================
% Layout Variablen
%--------------------
\index{Section!ServerLayout}
\subsection{Layout Variables: Section ServerLayout}
\begin{tabular}[h]{|p{2cm}|p{4cm}|p{7cm}|}
 \hline
 \textbf{Identifier} & \textbf{Variable}     & \textbf{Format}        \\
 \hline
 Integer & Identifier                 & String                   \\
 Integer & Keyboard                   & String                   \\
 Integer & InputDevice                & String,String,..         \\
 Integer & Xinerama                   & On \verb+|+ Off          \\
 Integer & Screen:\verb+<Identifier>+ & Left Right Up Down  \\
 \hline
\end{tabular}

\newpage

%====================
% Tastatur Variablen
%--------------------
\subsection{Keyboard Variables: InputDevice Section}
\begin{tabular}[h]{|p{2cm}|p{4cm}|p{7cm}|}
 \hline
 \textbf{Identifier} & \textbf{Variable}     & \textbf{Format}        \\
 \hline
 Integer & Identifier       & String                             \\
 Integer & Driver           & String                             \\
 Integer & Protocol         & String                             \\
 Integer & XkbRules         & String                             \\
 Integer & XkbModel         & String                             \\
 Integer & XkbLayout        & String                             \\
 Integer & XkbVariant       & String                             \\
 Integer & XkbOptions       & String,String,...                  \\
 Integer & AutoRepeat       & String                             \\
 Integer & Xleds            & String                             \\
 Integer & XkbDisable       & Yes \verb+|+ None                  \\
 Integer & VTSysReq         & Yes \verb+|+ None                  \\
 Integer & VTInit           & String                             \\
 Integer & ServerNumLock    & Yes \verb+|+ None                  \\
 Integer & LeftAlt          & String                             \\
 Integer & RightAlt         & String                             \\
 Integer & ScrollLock       & String                             \\
 Integer & RightCtl         & String                             \\
 Integer & XkbKeyCodes      & String                             \\
 \hline
\end{tabular}


\newpage
\index{API file examples}
\section{Example of API Variables}
\begin{small}
\begin{verbatim}
Keyboard {
 0 Protocol         =    Standard
 0 XkbLayout        =    de
 0 Identifier       =    Keyboard[0]
 0 XkbModel         =    pc105
 0 XkbVariant       =    nodeadkeys
 0 Driver           =    kbd
}
Mouse {
 1 ZAxisMapping     =    4 5
 1 Driver           =    mouse
 1 Device           =    /dev/input/mice
 1 ButtonNumber     =    5
 1 Vendor           =    Sysp
 1 Identifier       =    Mouse[1]
 1 Name             =    Autodetection
 1 Protocol         =    imps/2
}
Card {
 0 Name             =    RivaTNT
 0 Identifier       =    Device[0]
 0 BusID            =    1:0:0
 0 Driver           =    nv
 0 Vendor           =    Nvidia
}
Desktop {
 0 VertRefresh      =    50-160
 0 Device           =    Device[0]
 0 ModelName        =    Vision Master Pro 450 (A901HT)
 0 CalcModelines    =    yes
 0 Identifier       =    Screen[0]
 0 ColorDepth       =    16
 0 Monitor          =    Monitor[0]
 0 Modes:16         =    1800x1350,640x480
 0 HorizSync        =    27-115
 0 VendorName       =    Iiyama
}
Path {
 0 ServerFlags      =    AllowMouseOpenFail
 0 FontPath         =    /usr/lib/X11/fonts/misc:unscaled
 0 ModuleLoad       =    dbe,freetype,extmod,glx
}
Layout {
 0 Screen:Screen[0] =    <none> <none> <none> <none>
 0 InputDevice      =    Mouse[1]
 0 Keyboard         =    Keyboard[0]
 0 Identifier       =    Layout[all]
}
Extensions {
}
\end{verbatim}
\end{small}
